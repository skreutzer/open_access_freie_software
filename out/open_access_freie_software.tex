\documentclass[a4paper]{memoir}
% This file was generated by layout1 for html2latex1, which is free software licensed under the GNU Affero General Public License 3 or any later version (see https://github.com/publishing-systems/automated_digital_publishing/ and http://www.publishing-systems.org).

\usepackage[utf8]{inputenc}
\usepackage[T1]{fontenc}
\usepackage{lmodern}
\usepackage{ngerman}
\usepackage{url}

\setlength{\parskip}{0pt}

\renewcommand{\theenumi}{\arabic{enumi}}
\renewcommand{\labelenumi}{\theenumi}
\renewcommand{\theenumii}{\arabic{enumii}}
\renewcommand{\labelenumii}{\theenumii}
\renewcommand{\theenumiii}{\arabic{enumiii}}
\renewcommand{\labelenumiii}{\theenumiii}
\renewcommand{\theenumiv}{\arabic{enumiv}}
\renewcommand{\labelenumiv}{\theenumiv}

\renewcommand*{\maketitle}{%
  \thispagestyle{empty}
  \begingroup%
    \vspace*{\baselineskip}
    \vfill
    \hbox{%
      \vbox{%
        \vspace{0.1\textheight}
        {\noindent\HUGE\bfseries\centering Open Access, Freie Software und Co.: Eine Analyse der Gemengelage\par}\vspace*{5\baselineskip}
        \vspace{0.35\textheight} 
      }%
    }%
    \vfill
    \null
  \endgroup%
  \pagebreak{}
  \newpage{}
  \thispagestyle{empty}
  \mbox{}
  \pagebreak{}
}
    
\begin{document}
\frontmatter
\maketitle
\tableofcontents
\mainmatter
\chapter{Vier Konzepte}
Dieser Beitrag vergleicht den \textit{Open Access} mit der \textit{Freien Software} und ähnlichen Konzepten. Sein Ziel ist es, Parallelen und Unterschiede aufzuzeigen.

Da die Freie Software bereits seit den 80er Jahren als Konzept etabliert ist, der Open Access aber erst zwanzig Jahre später aufkam, können, so die Vermutung, aktuelle und zukünftige Entwicklungen beim Open Access nachvollzogen oder sogar vorweg erahnt werden, wenn man sich anschaut, wie sich die Freie Software entwickelt hat.

Dass im Titel der Begriff \textit{Freie Software} und nicht \textit{Open Source} vorkommt, wenn auch nur der Begriffsanalogie wegen, ist durchaus Absicht. Die Begriffe sind nicht so austauschbar, wie sie erscheinen mögen. Beide Bewegungen werden in diesem Beitrag behandelt. Daneben wird auch die Free Cultural Works-Bewegung betrachtet, die nach einer großen und lebendigen Allmende strebt. Die vier Konzepte und Bewegungen sind jeweils unterschiedliche Ausprägungen eines ähnlichen Gedankens, nämlich des \textit{Free Contents}.



\chapter{Hintergründe}
Um Konzepte und Bewegungen zu verstehen, muss man sich ihre Entstehungsgeschichten und ihre Strukturen anschauen.

\section{Freie Software}
Die Freie Software ist in erster Linie eine ethische und politische Bewegung, bei der die \textit{Rechte} der Menschen im Mittelpunkt stehen. Das wiederkehrende Leitbild ist der Wunsch, seinem Nachbarn etwas Gutes tun zu können. Dies soll ermöglicht werden. Deshalb soll Software frei sein.

Die Freie Software entstand in den 80er Jahren. Bis dahin war alle Software „frei“. Sie war damals eine Beigabe zur Hardware. Mit dem Beginn der 80er Jahre begannen Unternehmen, in Software eine Ware zu sehen, mit der man Geld verdienen kann.\textbackslash{}autocite[S. 13]\{spiegel06\} Statt sie kostenlos mitsamt dem Quellcode der Hardware beizulegen, wie zuvor, wurden die Programme, von da an, immer häufiger verkauft und ihr Quellcode geheim gehalten. Software wurde damit zu einem Produkt, das jemandem gehört. (Der passende Begriff für unfreie Software ist deshalb „proprietäre Software“,\textbackslash{}autocite[S. 28]\{spiegel06\} nicht aber „kommerzielle Software“.)

Die Freie Software entstand daraufhin als Gegenbewegung, wobei sie jedoch nicht den bisherigen Zustand abschaffen, sondern ihn beibehalten wollte. Der unbeschränkte Austausch von Software in Quellcodeform sollte erhalten bleiben. Die Freie Software ist demnach in ihrem Kern von bewahrendem Charakter. Sie stellte sich den neu aufkommenden Entwicklungen der damaligen Zeit, die heute zum Normalfall geworden sind, entgegen.

Wenn auch die Vorstellung, Software solle frei sein, in Programmiererkreisen weit verbreitet war, so war es Richard M. Stallman, der fast im Alleingang eine aktive Bewegung daraus machte. Sie manifestierte sich insbesondere im Start des GNU-Projekts (1983), in der Gründung der Free Software Foundation (1985) und im Verfassen der General Public License (1989), die alle von Stallman initiiert und vorangetrieben wurden.

Die Kultur des freien Austauschs von Information und Software entstammt primär dem universitären Umfeld. Stallman selbst war am Massachusetts Institute of Technology (MIT) verwurzelt. An der Westküste der USA, v. a. an der University of California, gab es ähnliche Bewegungen. Der ethische Fokus der Freien Software und damit seine politische Ausrichtung, die Stallman vertrat, waren jedoch in Kalifornien weniger präsent.

Die Grundmotivation der Freien Software ist die ethische Ansicht, dass Software keine Ware sein sollte, die jemandem gehört, sondern ein Gemeingut, das allen zur Verfügung steht. Die Analogie dazu sind Kochrezepte, die ganz natürlich weitergegeben, nachgekocht und abgewandelt werden können. 

\section{Open Source}
Open Source, wenn auch ähnlich zur Freien Software, hat eine andere Ausrichtung. Sie schätzt vor allem die verbesserten Möglichkeiten und die daraus resultierenden Konsequenzen, die einem offen stehen, wenn der Quellcode von Software zur Verfügung steht und dieser kopiert, verändert und verbreitet werden darf. Die Grundmotivation ist damit pragmatischer Natur.

Mitte der 90er Jahre, nachdem Linux, der Kernel, verfügbar war, als das Web sich verbreitete und Netscape im Browserkampf gegen Microsoft zu verlieren begann, sahen immer mehr Freie Software-Befürworter Probleme an dem Begriff „Freie Software“ und an seiner Ausrichtung. Das lag daran, dass das Wort „frei“ (im Deutschen wie im Englischen) zweideutig ist. Auch Stallmans regelmäßige Aufklärung – „Free software is a matter of liberty, not price. To understand the concept, you should think of free as in free speech, not as in free beer.“\textbackslash{}autocite\{fsf-def\} – löste dieses Problem nicht.\textbackslash{}autocite[S. 161f.]\{williams02\} Folglich wollte das kommerzielle Softwarebusiness nicht auf das Konzept aufspringen; zu stark war die Assoziation zu „gratis“, wenn auch die Freie Software nie gegen eine kommerzielle Verwertung war, sie ja sogar befürwortet.\textbackslash{}autocite\{selling-fs\} („‚Free software‘ does not mean ‚noncommercial‘. A free program must be available for commercial use, commercial development, and commercial distribution. Commercial development of free software is no longer unusual; such free commercial software is very important.“\textbackslash{}autocite\{fsf-def\}) Aber das Image passte dennoch nicht, wegen der Zweideutigkeit des Wortes „frei“. In dem Bestreben, die Freie Software auch im traditionellen Softwarebusiness zu verankern, trafen sich 1998 verschiedene Freie Software-Vertreter, um einen neuen, wirtschaftsfreundlicheren Begriff zu finden. Das Ergebnis war die Bezeichnung „Open Source“.\textbackslash{}autocite[S. 162f.]\{williams02\}

Stallman war zu diesem „Kick-off-Meeting“ nicht eingeladen, da er als zu starrköpfig und kompromisslos galt. Das Ziel der Beteiligten war auch gerade eine Umorientierung, weg von der ethischen und politischen Ausrichtung der Freien Software, die Stallman mit Überzeugung vertrat. Mit der pragmatischen, unpolitischen Ausrichtung der Open Source-Bewegung und der Ausgrenzung von Stallman spaltete sich die Gemeinschaft anschließend teilweise. Die eine Gruppe hielt weiterhin am Begriff „Freie Software“ fest und stand für die ethischen Ziele ein; die andere Gruppe nannte es „Open Source“ und legte auf die technischen Aspekte wert. Diese ideologische Spaltung war jedoch, und ist noch immer, kein Hindernis der gemeinsamen Arbeit, der Kooperation und des Austausches. (Neuere Bezeichnungen wie FLOSS, für „Free, Libre, and Open Source Software“, zeigen eine wiedervereinigende Motivation, wenn sie auch von vielen kritisch gesehen werden.\textbackslash{}autocite\{floss-foss\})

Die Open Source-Bewegung hängt weit weniger an einer einzelnen Person und den von ihr ausgehenden Organisationen und Projekten als die Freie Software mit Stallman. Der Evangelist der Open Source-Bewegung ist Eric S. Raymond. Zusammen mit Bruce Perens hat er 1998 die \textit{Open Source Initiative} (OSI) gegründet. Linus Torvalds, der den Kernel entwickelt hat, und Tim O'Reilly, der Verleger, gehören aber ebenso zu den Vertretern wie inzwischen auch große Softwareunternehmen. Open Source wird heutzutage durchaus businessfreundlich wahrgenommen.

Die Grundmotivation für Open Source ist die Ansicht, dass dieses Entwicklungsmodell zu besserer Software führt. Durch die freie Verfügbarkeit von Komponenten sowie durch offene Dokumentation und offenen Code würden Entwickler schneller und besser arbeiten können. Die Mitarbeit von Interessierten würde gefördert werden. Die relevanten Nutzerwünsche würden schneller umgesetzt werden. Angepasste Varianten würde eher entstehen. Die Ergebnisse würden sich schneller verbreiten. Fehler und Sicherheitslücken würden durch die freie Einsichtnahme in den Code schneller gefunden und behoben werden. Ob dem tatsächlich so ist und in welchen Fällen, bleibt weiterhin umstritten.

\section{Free Cultural Works}
Mit den Free Cultural Works (FCW)\textbackslash{}autocite\{fcw-def\} soll nun die Brücke von Software zu anderen Werken, darunter wissenschaftliche Publikationen, geschlagen werden. Bei den Free Cultural Works steht die Gemeinschaft und deren Allmende (das Gemeingut) im Zentrum. Werke sollen der Gemeinschaft gehören, nicht einzelnen Individuen. Ziel ist es, eine möglichst große Allmende aufzubauen, um so eine lebendige Kultur zu fördern.

Diese Bewegung ist weit weniger bekannt und weniger abgegrenzt als die anderen hier vorgestellten. Sie soll hier als ein konkreter Vertreter einer Vielzahl von ähnlichen Bewegungen, die allesamt die Allmende stärken wollen, auftreten.

Free Cultural Works wurde 2006 von Erik Möller, mit Unterstützung von Richard Stallman, Lawrence Lessig und weiteren ins Leben gerufen. Sie versuchten, im Kontext von Wikimedia, einen Standard zu legen, was als „Free Content“ angesehen werden kann. Der Nutzen der Free Cultural Works liegt darin, die heterogene Vielzahl von Lizenzen für intellektuelle und kreative Werke nach einem klaren Freiheitsstandard zu klassifizieren. Seit 2008 ist das bei den Creative Commons-Lizenzen der Fall: Nur zwei der sechs CC-Lizenzen (und der Public Domain Dedication CC0) ist die Erzeugung von Free Cultural Works bescheinigt. Desweiteren vermitteln sie ein Bewusstsein für die Freiheit von Werken. Wie auch bei der Freien Software stehen Free Cultural Works nicht gegen die kommerzielle Verwertung, wohl aber gegen das Eigentum an kulturellen Werken.

\section{Open Access}
Open Access ist ein Konzept des wissenschaftlichen Publikationswesens. Er hat im Kern das Streben nach dem Zugang zu Information. Es geht dabei darum, das Wissen aufzunehmen und sich darauf berufen zu können. Die Wissenschaft soll nicht von dem von ihr selbst erzeugten Wissen ausgeschlossen werden.

Der Open Access entstand als Antwort auf die Zeitschriftenkrise der 90er Jahre. Er kam v. a. in den STM-Wissenschaften auf, da dort Zeitschriftenartikel die Hauptpublikationsform darstellen. Open Access soll eine Alternative zu den immer teurer werdende Zeitschriftenabonnements, die zunehmend größere Teile der Wissenschaftswelt den Zugang zum publizierten Wissen erschweren, bieten. Im gleichen Zug spielt die Unzufriedenheit der Autoren über die zumeist exklusiv abzutretenden Rechten an ihren Werken mit. Auch die Frage, wie es um die Notwendigkeit von Verlagen bestellt ist, wo das Internet und umso mehr das Web mit Repositorien und Kommunikationskanälen ähnliche Verbreitungsmöglichkeiten ohne Rechteabtritt und quasi kostenlos bietet, steht im Raum. 

Im Gegensatz zur Entstehung der Freien Software, wo der Status Quo beibehalten werden sollte, geht es beim Open Access darum, eine Neuordnung der Situation zu erreichen. Diese Neuordnung wurde durch das Web, wo jeder selbst Verleger sein kann, ermöglicht. Wo die Freie Software von einer einzelnen Person, Richard Stallman, vorangetrieben wird, und beim Open Source eine gemeinsame Linie vorherrscht, gibt es beim Open Access eine Menge heterogener Akteure. So existiert auch keine von allen anerkannte, klare Definition des Begriffs, sondern eine Vielzahl von großteils impliziten oder schwammigen Definitionen.

Die zwei etablierten Open Access-Wege – der Grüne und der Goldene – sollen hier nur kurz erwähnt werden, denn sie beschreiben \textit{Umsetzungen} des Konzeptes, nicht aber das Konzept selbst. Bei ihnen geht es um finanzielle Aspekte und den Ort der Veröffentlichung. Für diesen Beitrag sind sie nebensächlich.

Open Access entspricht insofern der Ausrichtung von Open Source, da es auch darin primär um pragmatische Aspekte geht. Der Wunsch der Wissenschaftler ist es, schnell, einfach und kostenlos auf wissenschaftliche Erkenntnisse zugreifen zu können, die konkrete Rechtesituation oder gar der ethische Aspekt freien Wissens stehen im Hintergrund. Bei Open Source ist jedoch ein deutlich stärkeres Bewusstsein für eine klare Definition, Rechtslage und Einheitlichkeit vorhanden. Dies liegt wohl zum einen am Charakter seiner Beteiligten, die als Programmierer genaue Definitionen schätzen, als auch an ihrer Geburt aus der Freien Software, die eine klare Rechtslage als eine Kernaufgabe sieht. Nicht zuletzt ermöglichen auch anerkannte Leitfiguren eine Einigung auf klare Worte.

\chapter{Realisierungen}
Dieser Abschnitt stellt die Definitionen der verschiedenen Konzepte und typische Lizenzen vor.

\section{Freie Software}
Für die Freie Software gibt es eine Definition der Free Software Foundation,\textbackslash{}autocite\{fsf-def\} die vier Freiheiten umfasst. Sind diese gegeben, dann wird ein Stück Software als frei angesehen:

\begin{itemize}
\setlength{\itemsep}{0pt}
\item The freedom to run the program, for any purpose (freedom 0).
\item The freedom to study how the program works, and change it so it does your computing as you wish (freedom 1). Access to the source code is a precondition for this.
\item The freedom to redistribute copies so you can help your neighbor (freedom 2).
\item The freedom to distribute copies of your modified versions to others (freedom 3). By doing this you can give the whole community a chance to benefit from your changes. Access to the source code is a precondition for this.
\end{itemize}
Die FSF pflegt eine Liste von Software-Lizenzen, die sie nach dieser Definition als frei ansieht.\textbackslash{}autocite\{fsf-licenses\} Die \textit{General Public License} (GPL)\textbackslash{}autocite\{gpl\} ist die typische Lizenz für die Freie Software-Bewegung. Sie basiert auf einem besonderen Konstrukt, dem \textit{Copyleft}.\textbackslash{}autocite\{copyleft\} Dieses erzwingt, dass jedes abgeleitete Werk wiederum unter der gleichen Lizenz stehen muss. Damit wird verhindert, dass ein Stück GPL-lizenzierter Code jemals auf eine Weise genutzt werden kann, die nicht jedermann gleichfalls zur Verfügung steht. Alle auf Copyleft-Werke aufbauenden Werke werden also wiederum Freie Software sein. Dieser Zwang wird von manchen als Einschränkung der individuellen Freiheit angesehen, von anderen dagegen als Sicherung der Freiheit aller.

\section{Open Source}
Die Open Source-Definition der Open Source Initiative\textbackslash{}autocite\{osi-def\} ist eine leicht abgewandelte Formulierung der Debian Free Software Guidelines,\textbackslash{}autocite\{dfsg\} welche für die GNU/Linux-Distribution \textit{Debian} entwickelt worden sind. Die Ausrichtung auf die Bedürfnisse einer Distribution, also eines Projektes, das verschiedene Programme sinnvoll zusammenstellt, geeignet anpasst und dann als „Sammelwerk“ verbreitet, sind klar zu erkennen. Die Definition ist folglich eine Checkliste, die Lizenzen durchlaufen müssen, damit die damit lizensierte Software in die Distribution aufgenommen werden kann. Gefordert werden:

\begin{itemize}
\setlength{\itemsep}{0pt}
\item Free Redistribution
\item Source Code
\item Derived Works
\item Integrity of The Author's Source Code
\item No Discrimination Against Persons or Groups
\item No Discrimination Against Fields of Endeavor
\item Distribution of License
\item License Must Not Be Specific to a Product
\item License Must Not Restrict Other Software
\item License Must Be Technology-Neutral
\end{itemize}
Eine präferierte Open Source-Lizenz gibt es nicht. Dem Charakter von Open Source entsprechen BSD-artige Lizenzen aber am besten. Der Kern deren Aussage lässt sich umgangsprachlich so zusammenfassen: Mache mit dieser Software was du willst, solange du sagst, wer sie geschrieben hat. Und erwarte keine Garantie oder Haftung für irgendwas.

Zum allergrößten Teil entsprechen sich die Definitionen der OSI und FSF bei der Frage, wie eine konkrete Lizenz klassifiziert wird: „The two definitions lead to the same result in practice, but use superficially different language to get there.“\textbackslash{}autocite\{osi-faq\}

\section{Free Cultural Works}
Inspiriert von der Definition von Freier Software erfordern Free Cultural Works folgende essentiellen Freiheiten:\textbackslash{}autocite\{fcw-def\}

\begin{itemize}
\setlength{\itemsep}{0pt}
\item The freedom to use and perform the work
\item The freedom to study the work and apply the information
\item The freedom to redistribute copies
\item The freedom to distribute derivative works
\end{itemize}
Daneben gibt es zusätzliche Anforderungen:

\begin{itemize}
\setlength{\itemsep}{0pt}
\item Availability of source data
\item Use of a free format
\item No technical restrictions
\item No other restrictions or limitations
\end{itemize}
Wenn auch keine weiteren Einschränkungen und Begrenzungen erlaubt sind, so gibt es bestimmte Einschränkungen, die akzeptabel sind, ohne die essentiellen Freiheiten zu beeinflussen:

In particular, requirements for attribution, for symmetric collaboration (i.e., „copyleft“), and for the protection of essential freedom are considered permissible restrictions.

Typische Lizenzen für Free Cultural Works sind die zwei Creative Commons-Lizenzen CC BY und CC BY-SA, sowie die Public Domain Dedication CC0. (Die anderen CC-Lizenzen sind unfrei im Sinne dieser Definition.)

Auch für Free Cultural Works gibt es eine Liste von Lizenzen, die den Anforderungen genügen.\textbackslash{}autocite\{fcw-licenses\}

\section{Open Access}
Eine singuläre, anerkannte Definition, wie es für die anderen Konzepte der Fall ist, gibt es für Open Access nicht. Über die Jahre entstanden allerlei Definitionen, die sich teilweise unterscheiden.

Die erste Definition, die den Begriff „Open Access“ verwendet hatte, war die \textit{Budapest Open Access Initiative}\textbackslash{}autocite\{budapest02\} in 2002. Sie definiert:

The literature that should be freely accessible online is that which scholars give to the world without expectation of payment. [...] By „open access“ to this literature, we mean its free availability on the public internet, permitting any users to read, download, copy, distribute, print, [...], or use them for any other lawful purpose, without financial, legal, or technical barriers other than those inseparable from gaining access to the internet itself. The only constraint on reproduction and distribution, and the only role for copyright in this domain, should be to give authors control over the integrity of their work and the right to be properly acknowledged and cited.

Ein Jahr später erschien die \textit{Berlin Declaration on Open Access to Knowledge in the Sciences and Humanities}:\textbackslash{}autocite\{berlin03\}

The author(s) and right holder(s) of such contributions grant(s) to all users a free, irrevocable, worldwide, right of access to, and a license to copy, use, distribute, transmit and display the work publicly and to make and distribute derivative works, in any digital medium for any responsible purpose, subject to proper attribution of authorship ([...]), as well as the right to make small numbers of printed copies for their personal use.

(Sie basiert stark, teilweise sogar im Wortlaut, auf dem \textit{Bethesda Statement on Open Access Publishing},\textbackslash{}autocite\{bethesda03\} ebenfalls von 2003.)

Hier sind abgeleitete Werke nun auch explizit beachtet. Über die Budapester Erklärung hinaus geht auch die Forderung, dass das Werk mitsamt aller Quellmaterialien in einem Repositorium veröffentlicht werden muss. Zudem unterscheidet man zwischen der digitalen und materiellen Vervielfältigung und Verbreitung. Das kann sicher als Zugeständnis an das Verlagswesen gewertet werden. Bei der Freien Software gibt es diese Unterscheidung nicht. Bei Open Source ist sie sogar explizit ausgeschlossen. Im Gegensatz zur Budapester Erklärung ist das Thema der Kosten nicht so prominent präsentiert. Das entspricht der Situation bei den Definitionen für Freie und Open Source Software – libre, nicht gratis.

Als typische Lizenzen für Open Access-Inhalte haben sich die Creative Commons-Lizenzen etabliert. In der Neuauflage der Budapester Empfehlungen von 2012 wird sogar explizit die CC BY-Lizenz empfohlen.\textbackslash{}autocite\{budapest12\} Die Tendenz zu CC BY scheint sich (zumindest für Zeitschriftenartikel) durchzusetzen. Daneben sind aber auch die anderen CC-Lizenzen (v. a. CC BY-NC, CC BY-ND und CC BY-NC-ND) verbreitet. Was die reinen Quelldaten angeht, so werden diese inzwischen zumeist unter CC0 veröffentlicht ... falls sie denn veröffentlicht werden.



\chapter{Diskussion}
\section{Freiheit}
Die verschiedenen Bewegungen scheiden sich an der Frage, was als wichtiger angesehen wird, die Freiheit der Information im Generellen oder ihr konkreter praktischer Wert zum aktuellen Zeitpunkt.

Die Freie Software-Bewegung legt größten Wert auf die Freiheit, denn in ihr sieht sie die Voraussetzung für alle anderen Bestrebungen. Bruce Perens, der 1998 die Open Source Initiative mitgegründet hatte, wandte sich ein Jahr später wieder davon ab und der Freien Software zu, da für ihn der Wert der Freiheit wichtiger erschien:\textbackslash{}autocite\{perens-fs\}

Most hackers know that Free Software and Open Source are just two words for the same thing. Unfortunately, though, Open Source has de-emphasized the importance of the freedoms involved in Free Software. It's time for us to fix that. We must make it clear to the world that those freedoms are still important, and that software such as Linux would not be around without them.

Die Neuauflage der Empfehungen der Budapest Open Access Initiative liefert im Bezug auf die Bedeutung der Freiheit eine Rangfolge in erfreulicher Klarheit: „[...] we recognize that gratis access is better than priced access, libre access is better than gratis access, and libre under CC-BY or the equivalent is better than libre under more restrictive open licenses.“\textbackslash{}autocite\{budapest12\} (Nur über die konkrete Empfehlung von CC BY und was hier „equivalent“ bedeutet lässt sich streiten.)

Kritisch am Open Access zu sehen ist die fortwährende \textbf{Abhängigkeit} von der Verwertungsindustrie. Diese favorisiert, verständlicherweise, den Goldenen Weg, welcher diese Abhängigkeit beibehält. Die Verwerter-unabhängige Zugänglichmachung auf dem Grünem Weg, geht als \textit{Zweit}veröffentlichung in das Verständnis der Wissenschaftler ein. Wie anders wäre die Ausgangsbasis, würden die Wissenschaftler die freien Repositorien als natürlichen ersten Veröffentlichungsort wählen und anschließend in einem Verlag zweitveröffentlichen! Zu abwegig scheint dieser Ansatz nicht zu sein, denn beispielsweise mit dem Preprint-Server ArXiv ist die Praxis in der Physik gar nicht so weit davon entfernt.

Die idealistischen Bewegungen versuchen stets Abhängigkeiten zu vermeiden, um ihre eigene \textbf{Entscheidungsfreiheit} zu bewahren. Dabei spielt die Zusammensetzung der Beteiligten eine Rolle. Wie groß ist der Anteil derjenigen, die aus einem inneren Bedürfnis heraus, meist in ihrer Freizeit, aktiv sind, und wie groß ist der Anteil jener, für die es ein Job zum Lebensunterhalt ist? Die erste Gruppe tut sich deutlich einfacher damit, ihren persönlichen Vorstellungen nachzugehen. Die zweite Gruppe befindet sich in der Abhängigkeit, immer auch Erwartungen von außen entsprechen zu müssen. Ihre Entscheidungsfreiheit ist schon von Beginn an beschränkt.

Die Bewegungen Freie Software, Open Source, und nicht zuletzt Free Cultural Works zeigen eine Form der \textbf{Selbstbestimmung} der Urheber, die der Open Access nicht erkennen lässt. Der Grund mag darin liegen, dass bei ersteren eine größere Bindung zum eigenen Werk vorliegt, als es bei den Wissenschaftler der Fall zu sein scheint. Die Angst, dass man das eigene Werk „verliert“, wenn man Verwertern exklusive Nutzungsrechte einräumt, scheint bei den Wissenschaftlern nicht allzu groß zu sein. Die Veröffentlichung wird scheinbar mehr als Mittel zum Zweck gesehen. Wo aber das eigene Werk hoch geschätzt wird, wird ein größeres Bewusstsein für die (Urheber-)Rechtslage vorhanden sein. Unter freien Lizenzen bleibt einem selbst sein Werk zwar nicht vorbehalten, man kann aber die Rechte daran auch nicht verlieren.

\section{Gemeingut}
Eine weitere Unterscheidung der Bewegungen lässt sich im Bezug auf die \textbf{Hauptzielgruppe} treffen: Geht es in erster Linie um die Interessen der Gemeinschaft oder um die Interessen der Einzelperson?

Alle vorgestellten Bewegungen haben die gesamte Menschheit im Blick, wenn auch mit unterschiedlich starkem Fokus darauf. Sind also Ausnahmen für Untergruppen, wie beispielsweise die Forschung und Lehre, akzeptabel oder nicht? Die Bewegungen, die ethische Gesichtspunkte vertreten, verneinen. Die pragmatischen Bewegungen sehen darin aber eine einfachere Durchsetzbarkeit und somit mittelfristige Vorteile. Ob durch das ungenutzte, weil ausgegrenzte Potenzial oder durch immer wieder neu zu erkämpfende Grenzbereiche langfristige Nachteile entstehen, bleibt zu klären. Bei der Freien Software und den Free Cultural Works ist klar: Zuerst dem Volk, dann den Verwertern. Entscheidend dabei ist aber, dass nichts gegen eine kommerzielle Verwertung spricht, nur darf dieses Bestreben die Rechte der Allgemeinheit nicht beschränken.

Ein schönes Beispiel für eine Verpflichtungserklärung der Menschheit gegenüber ist der \textit{Debian Social Contract}.\textbackslash{}autocite\{dsc\} Eine so klare und konkrete Erklärung der Wissenschaft der Menschheit gegenüber wäre ein wertvolles Leitbild für die Open Access-Bewegung. Die Open Access-Erklärungen enthalten zwar Leitbilder, diese sind aber leider allzu oft voll wolkiger Worthülsen. Verständlich ist das Bedürfnis, sich nicht festnageln lassen zu wollen, gerade das jedoch wäre ein wichtiger Schritt in Richtung Glaubwürdigkeit.

Die im Open Access verbreitete Tendenz zu \textbf{Non-Commercial-Einschränkungen} (NC) gibt es bei den anderen Bewegungen nicht. Dort sieht man in kommerziellen Angeboten einen Mehrwert, auf den man nicht verzichten will. Beim Open Access mag die Tendenz daher rühren, dass auch die Verwerter selbst in der Bewegung aktiv sind und sich dieses Marktfeld exklusiv reserviert halten wollen.

Das Bedürfnis, zu verhindern, dass sich Andere am eigenen Werk bedienen ohne etwas zurückzugeben, ist durchaus auch in den anderen Bewegungen vorhanden. Das Mittel der Wahl dagegen ist das Copyleft-Prinzip. Dieses lässt die kommerzielle Nutzung sehr wohl zu, stellt aber sicher, dass jeder die gleichen Möglichkeiten der kommerziellen Nutzung hat und dass jedes aufbauende Werk dem ursprünglichen Urheber (und jedem sonst) ebenfalls zur Verfügung steht.

Ob nun solche \textbf{Copyleft}-Lizenzen gut sind oder nicht, darüber ist sich die Gemeinschaft nicht einig. Beide Lizenztypen, die mit Copyleft (z. B. die GPL) und die ohne (z. B. die BSD-artigen), bestehen nebeneinander, und das schon seit dreißig Jahren. Es ist nicht abzusehen, dass eine Art die Oberhand gewinnen würde. Bei den Creative Commons-Lizenzen gibt es mit CC BY und CC BY-SA ein äquivalentes Paar. (Dort wird „Copyleft“ als „Share-alike“ bezeichnet.) Auch hier werden wahrscheinlich beide Arten nebeneinander, gut möglich für unterschiedliche Publikationsformen, fortbestehen, da sie unterschiedliche Vor- und Nachteile haben.

\section{Schlagkraft}
Ein großer Unterschied zwischen Open Access und den anderen Konzepten ist die Menge seiner unterschiedlichen Beteiligten. Während sich die anderen Konzepte um kleine Gruppen von ähnlich Denkenden herum aufbauen, ist der Open Access eine Bewegung, die sehr viele Personen, Institutionen und Unternehmen mit ihren eigenen, unterschiedlichen Interessen mitformen, ohne dass es eine klare Führung gäbe. Wenn auch von den Wissenschaftlern initiiert, wirken nun auch viele andere Akteure mit. Als Folge wird der Begriff „Open Access“ inzwischen fast wahllos verwendet. Die wissenschaftliche Gemeinschaft – falls es die gibt – hat keine Form der Abgrenzung und Reinhaltung ihres Konzeptes gefunden. Wie sollte sie auch, wo sie sich selbst noch nicht klar ist, welche Werte und Forderungen sie denn vertritt. Wo die anderen Bewegungen anerkannte Definitionen vorweisen können, gelingt dies dem Open Access nicht. Zu stark ist die systemimmanente \textbf{Heterogenität} der Wissenschaft. Zu schwer fällt es den Wissenschaftlern, sich zu organisieren, zumindest sich schlagkräftig und konsequenzbereit zu organisieren. Zu stark sind aber auch die Traditionen des Publizierens, mit der starken Einflussposition der Unternehmen. So sind es nun eben diese Unternehmen, die die Praxis des Open Access prägen und ausgestalten. Nach anfänglichen Startschüssen haben die Wissenschaftler heute die Kontrolle großteils aus der Hand gegeben. Von der Definition des Open Access bleibt als gemeinsamer Nenner letztlich nur der kostenlose (Lese-)Zugriff, also der Wortsinn des Begriffes selbst, übrig. Nur hierin sind sich alle Beteiligten einig.

Anders bei der Open Source-Bewegung: Als Microsoft mit seinem \textit{Shared Source}-Konzept auf den Open Source-Zug aufspringen wollte, wurde das als reine Nutznießerei ohne erkennbare Unterstützung des Kerngedankens der Open Source-Bewegung erkannt und verurteilt.\textbackslash{}autocite\{perens-stand-together\} Folglich wendete sich die Gemeinschaft ab. Diese aktive \textbf{Abgrenzung} von reinen Trittbrettfahrern, die die Integrität der Bewegung verwässern würden, fehlt dem Open Access bislang. Sie setzt allerdings ein gemeinsames Selbstverständnis voraus.

Leider herrscht bei den Wissenschaftlern oft ein \textbf{Pragmatismus} vor, der lediglich den Erträglichkeitslevel akzeptabel halten will. Der idealistische Wunsch der grundlegenden Verbesserung geht meist neben den pragmatischen Anforderungen des Alltags unter.

\section{Qualität}
Mit Bezug auf Open Source kann man für den Open Access argumentieren, dass die Offenlegung aller Forschungsdaten und der daraus entstehenden Publikationen zu besseren Forschungsergebnissen führen kann. Das sogar auf mehrerlei Weise: Man bietet anderen Forschern und sonstigen Interessierten die Möglichkeit, Fehler zu finden und weitere Erkenntnisse zu entdecken; es werden aufbauende und zusammenführende Arbeiten gefördert; und nicht zuletzt werden die Wissenschaftler, aufgrund der Gewissheit, nachprüfbar zu sein, sorgfältiger arbeiten. Diese Verbesserungen der wissenschaftlichen \textbf{Qualität} müssen nicht eintreten, sie sind aber wahrscheinlich. Nachteile durch die Offenlegung sind nur zu befürchten, wenn die wissenschaftliche Ethik und Selbstorganisation versagen. Das bisherige Zögern der Wissenschaft mag von einem fehlenden Selbstbewusstsein oder von zu starkem Herdentrieb stammen.

\section{Fazit}
Die in diesem Beitrag vorgestellten Konzepte zeigen Möglichkeiten, wie sich Ziele und Wünsche vertreten lassen, so dass nebenrangige Beteiligte weiterhin bestehen und wertschöpfend sein können, ohne die zentralen Interessen zu gefährden. Notwendig dafür ist eine Bewegung mit einem schlagkräftigen und akzeptierten Kern an Wortführern und eine breite Basis von sich einigen Anhängern. Diese muss klare Definitionen und Ausrichtungen vorgeben und dann das Konzept rein halten.

An sich ist die Wissenschaft mit dem Open Access auf einem noch guten Weg. Die vorhandenen Definitionen sind eine brauchbare Ausgangsbasis, die bereits Konsolidierungstendenzen aufweist. Auch ein Bewusstsein für die Situation und ihre Hintergründe wird zunehmend geschaffen, gerade auch von den Bibliotheken. Entscheidend ist aber, dass das Bemühen jetzt, wo die Verwerter einzuschwenken beginnen, nicht nachlässt. Noch ist nichts grundlegend geändert. Noch ist die Situation nicht gut, nur nicht mehr untragbar. Jetzt ist der Zeitpunkt, aktiv zu werden! Jetzt muss die Wissenschaft ihr Selbstverständnis bestätigen! Jetzt muss sie ihre Definition von Open Access klarer machen! Jetzt muss die wissenschaftliche Gemeinschaft an ihrer Selbstorganisation arbeiten! Open Access-Publikationen müssen geschätzt werden! Der Gemeinschaft vorenthaltene oder nur erschwert zugängliche Publikationen müssen benachteiligt werden! Das Geheimhalten von Forschungsdaten muss kritisiert werden! Was in der Berlin Declaration schon vor einem Jahrzehnt gefordert wurde, muss die Praxis werden! Die blinde Lobhudelei auf der Basis von naiven Kennzahlen muss aufhören!

Es reicht aber nicht, die Wissenschaftler nur zu „bestärken“ und Open Access-Veröffentlichungen „anzuerkennen“. Nein! Die Wissenschaft muss Open Access spürbar wertschätzen! Die Umsetzung steht der Wissenschaft frei. Sie muss sich nur selbst organisieren und dann ihre eigenen Werte leben.

\section{Public Domain Dedication}
Für mich selbstvertändlicherweise ist dieses Werk frei (libre), offen und transparent. Das fertige Dokument, sein Quellcode (in Latex) und seine Entstehungsgeschichte (im Versionskontrollsystem stehen jedermann vollumfänglich zur Verfügung\footnote{\url{http://marmaro.de/docs/bib/oa-fs/}}. Mittels CC0 1.0 Universell\footnote{\url{http://creativecommons.org/publicdomain/zero/1.0/}}\textit{ }verzichte ich weltweit auf alle urheberrechtlichen und verwandten Schutzrechte, soweit das gesetzlich möglich ist.



\chapter{Lizenzbestimmungen}
Alle Teile dieses Werkes mit Ausnahme des nachfolgenden Lizenztextes: Copyright (C) 2014 Markus Schnalke, lizenziert unter Creative Commons Attribution 4.0 International Public License über „Perspektive Bibliothek 3.2“ (2014), S. 44-60, doi: 10.11588/pb.2014.2.16806. Nachfolgender Lizenztext Copyright (C) 2009 Creative Commons, lizenziert unter CC0 Public Domain Dedication\footnote{\url{https://creativecommons.org/licenses/zero/1.0/legalcode}}.

\section{Creative Commons Attribution 4.0 International Public License}
By exercising the Licensed Rights (defined below), You accept and agree to be bound by the terms and conditions of this Creative Commons Attribution 4.0 International Public License (“Public License”). To the extent this Public License may be interpreted as a contract, You are granted the Licensed Rights in consideration of Your acceptance of these terms and conditions, and the Licensor grants You such rights in consideration of benefits the Licensor receives from making the Licensed Material available under these terms and conditions.

\subsection{Section 1 – Definitions.}
\begin{enumerate}
\setlength{\itemsep}{0pt}
\item \textbf{Adapted Material} means material subject to Copyright and Similar Rights that is derived from or based upon the Licensed Material and in which the Licensed Material is translated, altered, arranged, transformed, or otherwise modified in a manner requiring permission under the Copyright and Similar Rights held by the Licensor. For purposes of this Public License, where the Licensed Material is a musical work, performance, or sound recording, Adapted Material is always produced where the Licensed Material is synched in timed relation with a moving image.
\item \textbf{Adapter's License} means the license You apply to Your Copyright and Similar Rights in Your contributions to Adapted Material in accordance with the terms and conditions of this Public License.
\item \textbf{Copyright and Similar Rights} means copyright and/or similar rights closely related to copyright including, without limitation, performance, broadcast, sound recording, and Sui Generis Database Rights, without regard to how the rights are labeled or categorized. For purposes of this Public License, the rights specified in Section 2.2.1-2 are not Copyright and Similar Rights.
\item \textbf{Effective Technological Measures} means those measures that, in the absence of proper authority, may not be circumvented under laws fulfilling obligations under Article 11 of the WIPO Copyright Treaty adopted on December 20, 1996, and/or similar international agreements.
\item \textbf{Exceptions and Limitations} means fair use, fair dealing, and/or any other exception or limitation to Copyright and Similar Rights that applies to Your use of the Licensed Material.
\item \textbf{Licensed Material} means the artistic or literary work, database, or other material to which the Licensor applied this Public License.
\item \textbf{Licensed Rights} means the rights granted to You subject to the terms and conditions of this Public License, which are limited to all Copyright and Similar Rights that apply to Your use of the Licensed Material and that the Licensor has authority to license.
\item \textbf{Licensor} means the individual(s) or entity(ies) granting rights under this Public License.
\item \textbf{Share} means to provide material to the public by any means or process that requires permission under the Licensed Rights, such as reproduction, public display, public performance, distribution, dissemination, communication, or importation, and to make material available to the public including in ways that members of the public may access the material from a place and at a time individually chosen by them.
\item \textbf{Sui Generis Database Rights} means rights other than copyright resulting from Directive 96/9/EC of the European Parliament and of the Council of 11 March 1996 on the legal protection of databases, as amended and/or succeeded, as well as other essentially equivalent rights anywhere in the world.
\item \textbf{You} means the individual or entity exercising the Licensed Rights under this Public License. \textbf{Your} has a corresponding meaning.
\end{enumerate}
\subsection{Section 2 – Scope.}
\begin{enumerate}
\setlength{\itemsep}{0pt}
\item \textbf{License grant}.\begin{enumerate}
\setlength{\itemsep}{0pt}
\item Subject to the terms and conditions of this Public License, the Licensor hereby grants You a worldwide, royalty-free, non-sublicensable, non-exclusive, irrevocable license to exercise the Licensed Rights in the Licensed Material to: \begin{enumerate}
\setlength{\itemsep}{0pt}
\item reproduce and Share the Licensed Material, in whole or in part; and
\item produce, reproduce, and Share Adapted Material.
\end{enumerate}

\item Exceptions and Limitations. For the avoidance of doubt, where Exceptions and Limitations apply to Your use, this Public License does not apply, and You do not need to comply with its terms and conditions.
\item Term. The term of this Public License is specified in Section 6.1.
\item Media and formats; technical modifications allowed. The Licensor authorizes You to exercise the Licensed Rights in all media and formats whether now known or hereafter created, and to make technical modifications necessary to do so. The Licensor waives and/or agrees not to assert any right or authority to forbid You from making technical modifications necessary to exercise the Licensed Rights, including technical modifications necessary to circumvent Effective Technological Measures. For purposes of this Public License, simply making modifications authorized by this Section 2.1.4 never produces Adapted Material.
\item Downstream recipients. \begin{enumerate}
\setlength{\itemsep}{0pt}
\item Offer from the Licensor – Licensed Material. Every recipient of the Licensed Material automatically receives an offer from the Licensor to exercise the Licensed Rights under the terms and conditions of this Public License.
\item No downstream restrictions. You may not offer or impose any additional or different terms or conditions on, or apply any Effective Technological Measures to, the Licensed Material if doing so restricts exercise of the Licensed Rights by any recipient of the Licensed Material.
\end{enumerate}

\item No endorsement. Nothing in this Public License constitutes or may be construed as permission to assert or imply that You are, or that Your use of the Licensed Material is, connected with, or sponsored, endorsed, or granted official status by, the Licensor or others designated to receive attribution as provided in Section 3.1.1.1.1.
\end{enumerate}

\item \textbf{Other rights}.\begin{enumerate}
\setlength{\itemsep}{0pt}
\item Moral rights, such as the right of integrity, are not licensed under this Public License, nor are publicity, privacy, and/or other similar personality rights; however, to the extent possible, the Licensor waives and/or agrees not to assert any such rights held by the Licensor to the limited extent necessary to allow You to exercise the Licensed Rights, but not otherwise.
\item Patent and trademark rights are not licensed under this Public License.
\item To the extent possible, the Licensor waives any right to collect royalties from You for the exercise of the Licensed Rights, whether directly or through a collecting society under any voluntary or waivable statutory or compulsory licensing scheme. In all other cases the Licensor expressly reserves any right to collect such royalties.
\end{enumerate}

\end{enumerate}
\subsection{Section 3 – License Conditions.}
Your exercise of the Licensed Rights is expressly made subject to the following conditions.

\begin{enumerate}
\setlength{\itemsep}{0pt}
\item \textbf{Attribution}.\begin{enumerate}
\setlength{\itemsep}{0pt}
\item If You Share the Licensed Material (including in modified form), You must:\begin{enumerate}
\setlength{\itemsep}{0pt}
\item retain the following if it is supplied by the Licensor with the Licensed Material:\begin{enumerate}
\setlength{\itemsep}{0pt}
\item identification of the creator(s) of the Licensed Material and any others designated to receive attribution, in any reasonable manner requested by the Licensor (including by pseudonym if designated);
\item a copyright notice;
\item a notice that refers to this Public License;
\item a notice that refers to the disclaimer of warranties;
\item a URI or hyperlink to the Licensed Material to the extent reasonably practicable;
\end{enumerate}

\item indicate if You modified the Licensed Material and retain an indication of any previous modifications; and
\item indicate the Licensed Material is licensed under this Public License, and include the text of, or the URI or hyperlink to, this Public License.
\end{enumerate}

\item You may satisfy the conditions in Section 3.1.1 in any reasonable manner based on the medium, means, and context in which You Share the Licensed Material. For example, it may be reasonable to satisfy the conditions by providing a URI or hyperlink to a resource that includes the required information.
\item If requested by the Licensor, You must remove any of the information required by Section 3.1.1.1 to the extent reasonably practicable.
\item If You Share Adapted Material You produce, the Adapter's License You apply must not prevent recipients of the Adapted Material from complying with this Public License.
\end{enumerate}

\end{enumerate}
\subsection{Section 4 – Sui Generis Database Rights.}
Where the Licensed Rights include Sui Generis Database Rights that apply to Your use of the Licensed Material:

\begin{enumerate}
\setlength{\itemsep}{0pt}
\item for the avoidance of doubt, Section 2.1.1 grants You the right to extract, reuse, reproduce, and Share all or a substantial portion of the contents of the database;
\item if You include all or a substantial portion of the database contents in a database in which You have Sui Generis Database Rights, then the database in which You have Sui Generis Database Rights (but not its individual contents) is Adapted Material; and
\item You must comply with the conditions in Section 3.1 if You Share all or a substantial portion of the contents of the database.
\end{enumerate}
For the avoidance of doubt, this Section 4 supplements and does not replace Your obligations under this Public License where the Licensed Rights include other Copyright and Similar Rights.

\subsection{Section 5 – Disclaimer of Warranties and Limitation of Liability.}
\begin{enumerate}
\setlength{\itemsep}{0pt}
\item \textit{Unless otherwise separately undertaken by the Licensor, to the extent possible, the Licensor offers the Licensed Material as-is and as-available, and makes no representations or warranties of any kind concerning the Licensed Material, whether express, implied, statutory, or other. This includes, without limitation, warranties of title, merchantability, fitness for a particular purpose, non-infringement, absence of latent or other defects, accuracy, or the presence or absence of errors, whether or not known or discoverable. Where disclaimers of warranties are not allowed in full or in part, this disclaimer may not apply to You.}
\item \textit{To the extent possible, in no event will the Licensor be liable to You on any legal theory (including, without limitation, negligence) or otherwise for any direct, special, indirect, incidental, consequential, punitive, exemplary, or other losses, costs, expenses, or damages arising out of this Public License or use of the Licensed Material, even if the Licensor has been advised of the possibility of such losses, costs, expenses, or damages. Where a limitation of liability is not allowed in full or in part, this limitation may not apply to You.}
\item The disclaimer of warranties and limitation of liability provided above shall be interpreted in a manner that, to the extent possible, most closely approximates an absolute disclaimer and waiver of all liability.
\end{enumerate}
\subsection{Section 6 – Term and Termination.}
\begin{enumerate}
\setlength{\itemsep}{0pt}
\item This Public License applies for the term of the Copyright and Similar Rights licensed here. However, if You fail to comply with this Public License, then Your rights under this Public License terminate automatically.
\item Where Your right to use the Licensed Material has terminated under Section 6.1, it reinstates:\begin{enumerate}
\setlength{\itemsep}{0pt}
\item automatically as of the date the violation is cured, provided it is cured within 30 days of Your discovery of the violation; or
\item upon express reinstatement by the Licensor.
\end{enumerate}
For the avoidance of doubt, this Section 6.2 does not affect any right the Licensor may have to seek remedies for Your violations of this Public License.
\item For the avoidance of doubt, the Licensor may also offer the Licensed Material under separate terms or conditions or stop distributing the Licensed Material at any time; however, doing so will not terminate this Public License.
\item Sections 1, 5, 6, 7 and 8 survive termination of this Public License. 
\end{enumerate}
\subsection{Section 7 – Other Terms and Conditions.}
\begin{enumerate}
\setlength{\itemsep}{0pt}
\item The Licensor shall not be bound by any additional or different terms or conditions communicated by You unless expressly agreed.
\item Any arrangements, understandings, or agreements regarding the Licensed Material not stated herein are separate from and independent of the terms and conditions of this Public License.
\end{enumerate}
\subsection{Section 8 – Interpretation.}
\begin{enumerate}
\setlength{\itemsep}{0pt}
\item For the avoidance of doubt, this Public License does not, and shall not be interpreted to, reduce, limit, restrict, or impose conditions on any use of the Licensed Material that could lawfully be made without permission under this Public License.
\item To the extent possible, if any provision of this Public License is deemed unenforceable, it shall be automatically reformed to the minimum extent necessary to make it enforceable. If the provision cannot be reformed, it shall be severed from this Public License without affecting the enforceability of the remaining terms and conditions.
\item No term or condition of this Public License will be waived and no failure to comply consented to unless expressly agreed to by the Licensor.
\item Nothing in this Public License constitutes or may be interpreted as a limitation upon, or waiver of, any privileges and immunities that apply to the Licensor or You, including from the legal processes of any jurisdiction or authority.
\end{enumerate}
\end{document}
